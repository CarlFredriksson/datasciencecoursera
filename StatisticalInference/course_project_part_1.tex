\documentclass[]{article}
\usepackage{lmodern}
\usepackage{amssymb,amsmath}
\usepackage{ifxetex,ifluatex}
\usepackage{fixltx2e} % provides \textsubscript
\ifnum 0\ifxetex 1\fi\ifluatex 1\fi=0 % if pdftex
  \usepackage[T1]{fontenc}
  \usepackage[utf8]{inputenc}
\else % if luatex or xelatex
  \ifxetex
    \usepackage{mathspec}
  \else
    \usepackage{fontspec}
  \fi
  \defaultfontfeatures{Ligatures=TeX,Scale=MatchLowercase}
\fi
% use upquote if available, for straight quotes in verbatim environments
\IfFileExists{upquote.sty}{\usepackage{upquote}}{}
% use microtype if available
\IfFileExists{microtype.sty}{%
\usepackage{microtype}
\UseMicrotypeSet[protrusion]{basicmath} % disable protrusion for tt fonts
}{}
\usepackage[margin=1in]{geometry}
\usepackage{hyperref}
\hypersetup{unicode=true,
            pdftitle={Statistical Inference: Exploring the Exponential Distribution},
            pdfauthor={Carl Fredriksson},
            pdfborder={0 0 0},
            breaklinks=true}
\urlstyle{same}  % don't use monospace font for urls
\usepackage{color}
\usepackage{fancyvrb}
\newcommand{\VerbBar}{|}
\newcommand{\VERB}{\Verb[commandchars=\\\{\}]}
\DefineVerbatimEnvironment{Highlighting}{Verbatim}{commandchars=\\\{\}}
% Add ',fontsize=\small' for more characters per line
\usepackage{framed}
\definecolor{shadecolor}{RGB}{248,248,248}
\newenvironment{Shaded}{\begin{snugshade}}{\end{snugshade}}
\newcommand{\KeywordTok}[1]{\textcolor[rgb]{0.13,0.29,0.53}{\textbf{#1}}}
\newcommand{\DataTypeTok}[1]{\textcolor[rgb]{0.13,0.29,0.53}{#1}}
\newcommand{\DecValTok}[1]{\textcolor[rgb]{0.00,0.00,0.81}{#1}}
\newcommand{\BaseNTok}[1]{\textcolor[rgb]{0.00,0.00,0.81}{#1}}
\newcommand{\FloatTok}[1]{\textcolor[rgb]{0.00,0.00,0.81}{#1}}
\newcommand{\ConstantTok}[1]{\textcolor[rgb]{0.00,0.00,0.00}{#1}}
\newcommand{\CharTok}[1]{\textcolor[rgb]{0.31,0.60,0.02}{#1}}
\newcommand{\SpecialCharTok}[1]{\textcolor[rgb]{0.00,0.00,0.00}{#1}}
\newcommand{\StringTok}[1]{\textcolor[rgb]{0.31,0.60,0.02}{#1}}
\newcommand{\VerbatimStringTok}[1]{\textcolor[rgb]{0.31,0.60,0.02}{#1}}
\newcommand{\SpecialStringTok}[1]{\textcolor[rgb]{0.31,0.60,0.02}{#1}}
\newcommand{\ImportTok}[1]{#1}
\newcommand{\CommentTok}[1]{\textcolor[rgb]{0.56,0.35,0.01}{\textit{#1}}}
\newcommand{\DocumentationTok}[1]{\textcolor[rgb]{0.56,0.35,0.01}{\textbf{\textit{#1}}}}
\newcommand{\AnnotationTok}[1]{\textcolor[rgb]{0.56,0.35,0.01}{\textbf{\textit{#1}}}}
\newcommand{\CommentVarTok}[1]{\textcolor[rgb]{0.56,0.35,0.01}{\textbf{\textit{#1}}}}
\newcommand{\OtherTok}[1]{\textcolor[rgb]{0.56,0.35,0.01}{#1}}
\newcommand{\FunctionTok}[1]{\textcolor[rgb]{0.00,0.00,0.00}{#1}}
\newcommand{\VariableTok}[1]{\textcolor[rgb]{0.00,0.00,0.00}{#1}}
\newcommand{\ControlFlowTok}[1]{\textcolor[rgb]{0.13,0.29,0.53}{\textbf{#1}}}
\newcommand{\OperatorTok}[1]{\textcolor[rgb]{0.81,0.36,0.00}{\textbf{#1}}}
\newcommand{\BuiltInTok}[1]{#1}
\newcommand{\ExtensionTok}[1]{#1}
\newcommand{\PreprocessorTok}[1]{\textcolor[rgb]{0.56,0.35,0.01}{\textit{#1}}}
\newcommand{\AttributeTok}[1]{\textcolor[rgb]{0.77,0.63,0.00}{#1}}
\newcommand{\RegionMarkerTok}[1]{#1}
\newcommand{\InformationTok}[1]{\textcolor[rgb]{0.56,0.35,0.01}{\textbf{\textit{#1}}}}
\newcommand{\WarningTok}[1]{\textcolor[rgb]{0.56,0.35,0.01}{\textbf{\textit{#1}}}}
\newcommand{\AlertTok}[1]{\textcolor[rgb]{0.94,0.16,0.16}{#1}}
\newcommand{\ErrorTok}[1]{\textcolor[rgb]{0.64,0.00,0.00}{\textbf{#1}}}
\newcommand{\NormalTok}[1]{#1}
\usepackage{graphicx,grffile}
\makeatletter
\def\maxwidth{\ifdim\Gin@nat@width>\linewidth\linewidth\else\Gin@nat@width\fi}
\def\maxheight{\ifdim\Gin@nat@height>\textheight\textheight\else\Gin@nat@height\fi}
\makeatother
% Scale images if necessary, so that they will not overflow the page
% margins by default, and it is still possible to overwrite the defaults
% using explicit options in \includegraphics[width, height, ...]{}
\setkeys{Gin}{width=\maxwidth,height=\maxheight,keepaspectratio}
\IfFileExists{parskip.sty}{%
\usepackage{parskip}
}{% else
\setlength{\parindent}{0pt}
\setlength{\parskip}{6pt plus 2pt minus 1pt}
}
\setlength{\emergencystretch}{3em}  % prevent overfull lines
\providecommand{\tightlist}{%
  \setlength{\itemsep}{0pt}\setlength{\parskip}{0pt}}
\setcounter{secnumdepth}{0}
% Redefines (sub)paragraphs to behave more like sections
\ifx\paragraph\undefined\else
\let\oldparagraph\paragraph
\renewcommand{\paragraph}[1]{\oldparagraph{#1}\mbox{}}
\fi
\ifx\subparagraph\undefined\else
\let\oldsubparagraph\subparagraph
\renewcommand{\subparagraph}[1]{\oldsubparagraph{#1}\mbox{}}
\fi

%%% Use protect on footnotes to avoid problems with footnotes in titles
\let\rmarkdownfootnote\footnote%
\def\footnote{\protect\rmarkdownfootnote}

%%% Change title format to be more compact
\usepackage{titling}

% Create subtitle command for use in maketitle
\providecommand{\subtitle}[1]{
  \posttitle{
    \begin{center}\large#1\end{center}
    }
}

\setlength{\droptitle}{-2em}

  \title{Statistical Inference: Exploring the Exponential Distribution}
    \pretitle{\vspace{\droptitle}\centering\huge}
  \posttitle{\par}
    \author{Carl Fredriksson}
    \preauthor{\centering\large\emph}
  \postauthor{\par}
    \date{}
    \predate{}\postdate{}
  

\begin{document}
\maketitle

\subsection{Overview}\label{overview}

In this part of the project we will do some simulations. The aim of the
simulations are to investigate the exponential distribution and compare
it with the Central Limit Theorem.

\subsubsection{Simulations}\label{simulations}

Let us start by defining some constants and setting the seed so the
simulations can be replicated.

\begin{Shaded}
\begin{Highlighting}[]
\NormalTok{lambda =}\StringTok{ }\FloatTok{0.2}
\NormalTok{n =}\StringTok{ }\DecValTok{40}
\NormalTok{num_simulations =}\StringTok{ }\DecValTok{1000}
\KeywordTok{set.seed}\NormalTok{(}\DecValTok{1}\NormalTok{)}
\end{Highlighting}
\end{Shaded}

We can now run simulations of sampling from the exponential distribution
and taking the mean for each sample.

\begin{Shaded}
\begin{Highlighting}[]
\NormalTok{sample_means =}\StringTok{ }\OtherTok{NULL}
\ControlFlowTok{for}\NormalTok{ (i }\ControlFlowTok{in} \DecValTok{1} \OperatorTok{:}\StringTok{ }\NormalTok{num_simulations) sample_means =}\StringTok{ }\KeywordTok{c}\NormalTok{(sample_means, }\KeywordTok{mean}\NormalTok{(}\KeywordTok{rexp}\NormalTok{(n, lambda)))}
\KeywordTok{head}\NormalTok{(sample_means)}
\end{Highlighting}
\end{Shaded}

\begin{verbatim}
## [1] 4.860372 5.961285 4.279204 4.702298 5.196446 4.397114
\end{verbatim}

Let us also simulate a large collection of random exponentials.

\begin{Shaded}
\begin{Highlighting}[]
\NormalTok{exponentials =}\StringTok{ }\OtherTok{NULL}
\ControlFlowTok{for}\NormalTok{ (i }\ControlFlowTok{in} \DecValTok{1} \OperatorTok{:}\StringTok{ }\NormalTok{num_simulations) exponentials =}\StringTok{ }\KeywordTok{c}\NormalTok{(exponentials, }\KeywordTok{rexp}\NormalTok{(n, lambda))}
\KeywordTok{head}\NormalTok{(exponentials)}
\end{Highlighting}
\end{Shaded}

\begin{verbatim}
## [1] 6.7905260 4.8846486 2.7681238 4.1226835 0.5983089 6.4075421
\end{verbatim}

\subsubsection{Sample Mean versus Theoretical
Mean}\label{sample-mean-versus-theoretical-mean}

Let us compute the theoretical mean and the sample mean.

\begin{Shaded}
\begin{Highlighting}[]
\NormalTok{m_p <-}\StringTok{ }\DecValTok{1} \OperatorTok{/}\StringTok{ }\NormalTok{lambda}
\NormalTok{m_s <-}\StringTok{ }\KeywordTok{mean}\NormalTok{(sample_means)}
\end{Highlighting}
\end{Shaded}

When we compare them we see that they are very similar.

\begin{Shaded}
\begin{Highlighting}[]
\NormalTok{m_p}
\end{Highlighting}
\end{Shaded}

\begin{verbatim}
## [1] 5
\end{verbatim}

\begin{Shaded}
\begin{Highlighting}[]
\NormalTok{m_s}
\end{Highlighting}
\end{Shaded}

\begin{verbatim}
## [1] 4.990025
\end{verbatim}

\begin{Shaded}
\begin{Highlighting}[]
\KeywordTok{barplot}\NormalTok{(}\KeywordTok{c}\NormalTok{(m_p, m_s), }\DataTypeTok{names.arg=}\KeywordTok{c}\NormalTok{(}\StringTok{"Theoretical (1/lambda)"}\NormalTok{, }\StringTok{"Sample"}\NormalTok{), }\DataTypeTok{main=}\StringTok{"Theoretical mean vs sample mean"}\NormalTok{)}
\end{Highlighting}
\end{Shaded}

\includegraphics{course_project_part_1_files/figure-latex/unnamed-chunk-5-1.pdf}

\subsubsection{Sample Variance versus Theoretical
Variance}\label{sample-variance-versus-theoretical-variance}

Let us compute the theoretical variance and the sample variance.

\begin{Shaded}
\begin{Highlighting}[]
\NormalTok{v_p <-}\StringTok{ }\NormalTok{(}\DecValTok{1} \OperatorTok{/}\StringTok{ }\NormalTok{lambda)}\OperatorTok{^}\DecValTok{2}\OperatorTok{/}\NormalTok{n}
\NormalTok{v_s <-}\StringTok{ }\KeywordTok{sd}\NormalTok{(sample_means)}\OperatorTok{^}\DecValTok{2}
\end{Highlighting}
\end{Shaded}

As with the means, when we compare the variances we see that they are
very similar.

\begin{Shaded}
\begin{Highlighting}[]
\NormalTok{v_p}
\end{Highlighting}
\end{Shaded}

\begin{verbatim}
## [1] 0.625
\end{verbatim}

\begin{Shaded}
\begin{Highlighting}[]
\NormalTok{v_s}
\end{Highlighting}
\end{Shaded}

\begin{verbatim}
## [1] 0.6111165
\end{verbatim}

\begin{Shaded}
\begin{Highlighting}[]
\KeywordTok{barplot}\NormalTok{(}\KeywordTok{c}\NormalTok{(v_p, v_s), }\DataTypeTok{names.arg=}\KeywordTok{c}\NormalTok{(}\StringTok{"Theoretical ((1/lambda)^2/n)"}\NormalTok{, }\StringTok{"Sample"}\NormalTok{), }\DataTypeTok{main=}\StringTok{"Theoretical variance vs sample variance"}\NormalTok{)}
\end{Highlighting}
\end{Shaded}

\includegraphics{course_project_part_1_files/figure-latex/unnamed-chunk-7-1.pdf}

\subsubsection{Distribution}\label{distribution}

Lets compare the sample means we simulated to the simulated
exponentials.

\begin{Shaded}
\begin{Highlighting}[]
\KeywordTok{par}\NormalTok{(}\DataTypeTok{mfrow=}\KeywordTok{c}\NormalTok{(}\DecValTok{1}\NormalTok{, }\DecValTok{2}\NormalTok{))}
\KeywordTok{hist}\NormalTok{(sample_means, }\DataTypeTok{main=}\StringTok{"Sample means"}\NormalTok{, }\DataTypeTok{xlab=}\StringTok{"Mean"}\NormalTok{)}
\KeywordTok{hist}\NormalTok{(exponentials, }\DataTypeTok{main=}\StringTok{"Exponentials"}\NormalTok{, }\DataTypeTok{xlab=}\StringTok{"Value"}\NormalTok{)}
\end{Highlighting}
\end{Shaded}

\includegraphics{course_project_part_1_files/figure-latex/unnamed-chunk-8-1.pdf}

We can see that the sample means are approximately normally distributed,
even though the underlying exponential distribution is skewed to the
right.


\end{document}
